% !Mode:: "TeX:UTF-8"
%% 请使用 XeLaTeX 编译本文.
\documentclass[forprint]{report}
%---------------------这里添加所需的package--------------------------------
\usepackage{url}
\usepackage{amsmath}
\usepackage{listings}
\usepackage[dvipsnames]{xcolor}
\usepackage{fancyhdr}
\usepackage{titlesec}

\lstset{
	basicstyle={\ttfamily\footnotesize},
	keywordstyle={\bfseries\color{NavyBlue}}, % 关键词风格(listing包定义的词,也可手动添加)
	emph={String,main}, % 设置强调词
	emphstyle={\bfseries\color{Rhodamine}}, % 强调词风格
	alsoletter={.},
	commentstyle={\sffamily\color{PineGreen!50!black}}, % 注释风格
	stringstyle={\rmfamily\color{Maroon}}, % 字符串风格
	frame=shadowbox, % 带阴影边框
	framesep=1em, % 边框距代码距离
	framerule=1.3pt, % 边框粗细
	backgroundcolor={\color{lightgray!20!white}}, % 背景颜色
	rulecolor={\color{CadetBlue}}, % 边框颜色
	rulesep=3pt, % 阴影大小
	rulesepcolor={\color{CadetBlue!40!white}}, % 阴影颜色
	numbers=left, % 行号位置
	numbersep=2em, % 行号与边框间距
	numberstyle={\scriptsize\color{darkgray}}, % 行号风格
	showstringspaces=false, % 不显示字符串内空格
	showspaces=false, % 不显示空格
	showtabs=false, % 不显示Tab
	tabsize=4, % 设置Tab大小
	breaklines=true, % 自动换行
}

%--------------------------------------------------------------------------
\makeatletter
\def\BState{\State\hskip-\ALG@thistlm}
\makeatother

\title{笔 记} % 报告主题
\author{李 子 奥} % 学生姓名
\Csupervisor{x x x} % 指导教师姓名
\CstudentNum{U 2 0 1 9 1 4 6 2 9} % 学号
\Cmajor{自 动 化 1 9 0 3 班} % 专业名称
\Cschoolname{人 工 智 能 与 自 动 化 学 院} % 学院名
\date{\today} % 日期

\pagestyle{myfancy}
\assignpagestyle{\chapter}{myfancy}

\begin{document}
\pdfbookmark[0]{封面}{title}         % 封面页加到 pdf 书签
\maketitle
\frontmatter
\pagenumbering{Roman}              % 正文之前的页码用大写罗马字母编号.正文之前的页码隐藏,无需显示
%==========================把目录加入到书签==============================%%%%%%

\tableofcontents
\thispagestyle{empty}				%不显示罗马数字
\addtocontents{toc}{\protect\thispagestyle{empty}}

\mainmatter %% 以下是正文
%%%%%%%%%%%%%%%%%%%%%%%%%%%--------main matter-------%%%%%%%%%%%%%%%%%%%%%%%%%%%%%%%%%%%%
\baselineskip=23pt  % 正文行距为 23 磅

\chapter{SVM}

\section{SVM简介}

支撑向量机(Support Vector Machine, SVM)是一种二分类模型\cite{r3},它的基本模型是定义在特征空间上的间隔最大的线性分类器。它与感知机的区别在于 SVM 使分类平面与样本点距离的最大。间隔最大化的问题可形式化为一解凸二次规划的问题,也等价于正则化的合页损失函数的最小化问题。


SVM 还可以利用核技巧,从而对非线性的数据进行分类。

\section{Hard-margin SVM}

将不同类别的样本点完全分到分类平面的两侧,但当有异常样本点出现的时候,硬间隔的 SVM 可能会将异常点作为支撑向量,导致分类平面不佳\cite{r1}...

\begin{lstlisting}[language=c]
#include <stdio.h>
int main()
{
	printf("Hello, World!");
	return 0;
}
\end{lstlisting}

将不同类别的样本点完全分到分类平面的两侧,但当有异常样本点出现的时候,硬间隔的 SVM 可能会将异常点作为支撑向量,导致分类平面不佳\cite{r1}...

\begin{lstlisting}[language=c]
	#include <stdio.h>
	int main()
	{
		printf("Hello, World!");
		return 0;
	}
\end{lstlisting}

将不同类别的样本点完全分到分类平面的两侧,但当有异常样本点出现的时候,硬间隔的 SVM 可能会将异常点作为支撑向量,导致分类平面不佳\cite{r1}...

\begin{lstlisting}[language=c]
	#include <stdio.h>
	int main()
	{
		printf("Hello, World!");
		return 0;
	}
\end{lstlisting}

将不同类别的样本点完全分到分类平面的两侧,但当有异常样本点出现的时候,硬间隔的 SVM 可能会将异常点作为支撑向量,导致分类平面不佳\cite{r1}...

\begin{lstlisting}[language=c]
	#include <stdio.h>
	int main()
	{
		printf("Hello, World!");
		return 0;
	}
\end{lstlisting}

\chapter{Adaboost}
	
What is Adaboost?



%%%=== 参考文献 ========%%%
\pagestyle{mybibli}
\cleardoublepage\phantomsection
\addcontentsline{toc}{chapter}{参考文献}
%\bibliography{example}
\renewcommand{\baselinestretch}{1.6}

\begin{thebibliography}{00}\thispagestyle{mybibli}

  \bibitem{r1} 深入浅出KNN算法(一) KNN算法原理 \url{https://www.cnblogs.com/listenfwind/p/10311496.html}.
  \bibitem{r1} 深入浅出KNN算法(一) KNN算法原理 \url{https://www.cnblogs.com/listenfwind/p/10311496.html}.
  \bibitem{r1} 深入浅出KNN算法(一) KNN算法原理 \url{https://www.cnblogs.com/listenfwind/p/10311496.html}.
  \bibitem{r1} 深入浅出KNN算法(一) KNN算法原理 \url{https://www.cnblogs.com/listenfwind/p/10311496.html}.
  \bibitem{r1} 深入浅出KNN算法(一) KNN算法原理 \url{https://www.cnblogs.com/listenfwind/p/10311496.html}.
  \bibitem{r1} 深入浅出KNN算法(一) KNN算法原理 \url{https://www.cnblogs.com/listenfwind/p/10311496.html}.
  \bibitem{r1} 深入浅出KNN算法(一) KNN算法原理 \url{https://www.cnblogs.com/listenfwind/p/10311496.html}.
  \bibitem{r1} 深入浅出KNN算法(一) KNN算法原理 \url{https://www.cnblogs.com/listenfwind/p/10311496.html}.
  \bibitem{r1} 深入浅出KNN算法(一) KNN算法原理 \url{https://www.cnblogs.com/listenfwind/p/10311496.html}.
  \bibitem{r1} 深入浅出KNN算法(一) KNN算法原理 \url{https://www.cnblogs.com/listenfwind/p/10311496.html}.
  \bibitem{r1} 深入浅出KNN算法(一) KNN算法原理 \url{https://www.cnblogs.com/listenfwind/p/10311496.html}.
  \bibitem{r1} 深入浅出KNN算法(一) KNN算法原理 \url{https://www.cnblogs.com/listenfwind/p/10311496.html}.
  \bibitem{r1} 深入浅出KNN算法(一) KNN算法原理 \url{https://www.cnblogs.com/listenfwind/p/10311496.html}.
  \bibitem{r1} 深入浅出KNN算法(一) KNN算法原理 \url{https://www.cnblogs.com/listenfwind/p/10311496.html}.
  \bibitem{r1} 深入浅出KNN算法(一) KNN算法原理 \url{https://www.cnblogs.com/listenfwind/p/10311496.html}.
  \bibitem{r1} 深入浅出KNN算法(一) KNN算法原理 \url{https://www.cnblogs.com/listenfwind/p/10311496.html}.

\end{thebibliography}

\cleardoublepage
\end{document}