% !Mode:: "TeX:UTF-8"
%% 请使用 XeLaTeX 编译本文.
 \documentclass[forprint]{report}
%---------------------这里添加所需的package--------------------------------
\usepackage{url}
\usepackage{amsmath}

%--------------------------------------------------------------------------
\makeatletter
\def\BState{\State\hskip-\ALG@thistlm}
\makeatother
\begin{document}
%-----------------------------------------------------------------------------

\title{笔记} % 报告主题
\author{李子奥} % 学生姓名
\Csupervisor{xxx} % 指导教师姓名
\CstudentNum{U201914629} % 学号
\Cmajor{自动化1903} % 专业名称
\Cschoolname{人工智能与自动化学院} % 学院名
\date{\today} % 日期

%-----------------------------------------------------------------------------

\pdfbookmark[0]{封面}{title}         % 封面页加到 pdf 书签
\maketitle
\frontmatter
\pagenumbering{Roman}              % 正文之前的页码用大写罗马字母编号.正文之前的页码隐藏,无需显示
%-----------------------------------------------------------------------------
%==========================把目录加入到书签==============================%%%%%%

\tableofcontents
\thispagestyle{empty}				%不显示罗马数字
\addtocontents{toc}{\protect\thispagestyle{empty}}


\mainmatter %% 以下是正文
%%%%%%%%%%%%%%%%%%%%%%%%%%%--------main matter-------%%%%%%%%%%%%%%%%%%%%%%%%%%%%%%%%%%%%
\pagestyle{plain}%plain
%\cfoot{\thepage{\zihao{5}\bf\usefonttimes}}
%\renewcommand{\baselinestretch}{1.6}
%\setlength{\baselineskip}{23pt}
\baselineskip=23pt  % 正文行距为 23 磅

%此处书写正文-------------------------------------------------------------------------------------
\chapter{SVM}

\section{SVM简介}

支撑向量机(Support Vector Machine, SVM)是一种二分类模型\cite{r3},它的基本模型是定义在特征空间上的间隔最大的线性分类器。它与感知机的区别在于 SVM 使分类平面与样本点距离的最大。间隔最大化的问题可形式化为一解凸二次规划的问题,也等价于正则化的合页损失函数的最小化问题。


SVM 还可以利用核技巧,从而对非线性的数据进行分类。

\section{Hard-margin SVM}

将不同类别的样本点完全分到分类平面的两侧,但当有异常样本点出现的时候,硬间隔的 SVM 可能会将异常点作为支撑向量,导致分类平面不佳\cite{r1}...


%此处结束正文-------------------------------------------------------------------------------------------------
%%%============================================================================================================%%%

%%%=== 参考文献 ========%%%
\cleardoublepage\phantomsection
\addcontentsline{toc}{chapter}{参考文献}
%\bibliography{example}
\renewcommand{\baselinestretch}{1.6}

\begin{thebibliography}{00}

  \bibitem{r1} 深入浅出KNN算法(一) KNN算法原理 \url{https://www.cnblogs.com/listenfwind/p/10311496.html}.

\end{thebibliography}

\cleardoublepage
\end{document}