\documentclass[10pt,UTF8]{ctexbeamer} 
%默认字体12pt,编码使用UTF8,使用中文

\usetheme[progressbar=frametitle]{metropolis} 
%使用metropolis这一beamer主题
\usepackage{appendixnumberbeamer}

\usepackage{booktabs} %书签
\usepackage[scale=2]{ccicons} %typeset Creative Commons icons

\usepackage{pgfplots}
\usepgfplotslibrary{dateplot}
\usepackage{hologo}

\usepackage{amsmath, amsfonts, amssymb} %数学符号

%语法高亮
\usepackage{minted}
\usepackage{listings}

\usepackage{}
\usepackage{hyperref}
\hypersetup{
	colorlinks=true,
	linkcolor=blue,
	filecolor=blue,      
	urlcolor=red,
	citecolor=cyan,
}

\usepackage{graphicx}
\DeclareGraphicsExtensions{.eps,.ps,.jpg,.bmp,.png,.svg}
%指定可以使用以上格式图像文件

\usepackage{xspace} %断行命令
\newcommand{\themename}{\textbf{\textsc{metropolis}}\xspace}

\title{Metropolis} %标题
\subtitle{一个优秀的beamer主题} %副标题
\date{\today} %今天的日期
% \date{} %指定日期
\author{超级懒的小周} %作者
\institute{texpage.com} %机构
\titlegraphic{\hfill\includegraphics[height=2.5cm]{image/academic.png}} %插入logo

%在每个section 前边单独插入当前章节高亮的目录页(当然最原始的目录页你还是需要手动录入的,不要想偷懒)
%\AtBeginSection[]
%{
%	\begin{frame}
%		\frametitle{当前章节}
%		\tableofcontents[currentsection]
%	\end{frame}
%}

\begin{document}

\maketitle %目录生成

\begin{frame}{目录}
	%		Table of contents 目录/大纲页
	%             自动实现对section的收集,并绘制成目录页
  \setbeamertemplate{section in toc}[sections numbered]
  \tableofcontents%[hideallsubsections]
\end{frame}

\section{What is beamer?}
\begin{frame}[allowframebreaks]{What is beamer?}
	Beamer 是一个用于创建演示文稿 LaTeX 的文档类。它同时支持\LaTeX + dvips、\hologo{pdfLaTeX}、\hologo{LuaLaTeX}以及\hologo{XeLaTeX}。它的名称取自德语词汇 Beamer(pseudo-anglicism),意思是影像演示。
	
	Beamer文档类并不是最早开发出的\LaTeX 演示文稿工具。2003年2月,Till Tantau为其博士论文答辩编写了beamer包,并于一个月之后发布在CTAN上。
	
	作为LaTeX的一个文档类,Beamer文档和\LaTeX 文档一样都是纯文本文件。且beamer兼容\LaTeX 常见的命令,和其他宏包的兼容性良好。
	
	当然也有支持Beamer语法的图形界面,如AUCTEX和LyX。
	
	Beamer也可以通过使用兼容包来支持其他LaTeX演示文稿宏包的语法,包括 Prosper和Foils。
	
	Beamer默认生成PDF文件用于演示,其动态效果依靠创建多页幻灯片实现。
	
	若要打印出每张幻灯片的最终效果用于分发给听众,需开启handout选项;想要在一张纸上打印多页幻灯片,需要用pgfpages宏包;也可以输出适合印刷在A4或者标准信纸上的文档效果。
	
	'frame'的标题将变为段落的标题,不再包括原有的外观主题,同时保证了原有章节结构不被破坏——这就可以方便的输出演讲的提纲。
	
	Beamer的一些功能是依赖于PGF的。
	
	以上介绍来自WiKipedia。\footnote{https://zh.wikipedia.org/wiki/Beamer\_(LaTeX)}
	
	关于如何自制一份beamer,请浏览\href{https://zhuanlan.zhihu.com/p/423443762}{用LaTeX创建一个Beamer},可以关注专栏,及时获得推动与建议。
\end{frame}

\section[Intro]{Introduction}

\begin{frame}[fragile]{Metropolis}
	
	\themename 主题是一款视觉噪音极小的 Beamer 主题,灵感来自 Benjamin Weiss 的 \href{https://github.com/hsrmbeamertheme/hsrmbeamertheme}{\textsc{hsrm} Beamer} 主题。
	
	启用这一主题通过加载
	
	\begin{verbatim}    
		\documentclass{beamer}
		\usetheme{metropolis}
	\end{verbatim}

   请注意,您必须安装 Mozilla's \emph{Fira Sans} 字体和 XeTeX 才能享受这种美妙的排版。
   
   \small{在 R 中,您当然可以直接使用这个包,请参阅它的文档。}
\end{frame}

\begin{frame}[fragile]{Sections}
  Sections 对同一主题的幻灯片进行分组

  \begin{verbatim}    
  	\section{Elements}
  \end{verbatim}

当然也有subsection对内容进行分节
  \begin{verbatim}
  	\subsection{subsection}
  \end{verbatim}
   这个 \themename 主题提供了一个很好的进度指示器 \ldots
  
\end{frame}

\section{提供讲义}
\begin{frame}[fragile]{如何提供一份讲义}
	你可以给你的观众一张幻灯片的印刷版。只需使用handout选项创建文档的一个版本,就不会使用覆盖图:
	\begin{verbatim}
		\documentclass[handout]{beamer}
		\usepackage{pgfpages}
		\pgfpagesuselayout{4 on 1}[a4paper,border shrink=5mm,landscape]
	\end{verbatim}
这将在A4纸上横向打印四张幻灯片。通过指定以下代码行,可以在纵向模式下获得更大的打印,每页两张幻灯片:
	\begin{verbatim}
		\pgfpagesuselayout{2 on 1}[a4paper,border shrink=5mm]
	\end{verbatim}
使用这一功能必须加载pgf宏包,并启用pgfpages选项,这是pgf宏包附带的一个实用宏包。
\end{frame}

\section{超链接}
\begin{frame}[fragile]{在beamer中设置超链接}
	可以这样
\begin{minted}{latex}
	\usepackage{hyperref}
	\hypersetup{
		colorlinks=true,
		linkcolor=blue,%%修改此处为你想要的颜色
		filecolor=blue,      
		urlcolor=red,
		citecolor=cyan,}
\end{minted}
也可以这样
\begin{minted}{latex}
	\usepackage[colorlinks,
	linkcolor=blue,       %%修改此处为你想要的颜色
	anchorcolor=blue,  
	citecolor=blue,       
	]{hyperref}
\end{minted}
更多请翻阅hyperref文档。
\end{frame}

\section{语法高亮}
\begin{frame}[fragile]{语法高亮的选择}
	\begin{itemize}
		\item 使用verbatim抄录环境:\verb|\verb|
		
		\item 使用minted环境:\mint{latex}{\mint}
		
		\item 使用listings环境:
		\begin{lstlisting}[language=c++]
			#include <iostream>
			int main()
		\end{lstlisting}
	\end{itemize}
其中verbatim是可以直接用的,minted和listings都需要在导言区加载宏包才可以使用,而minted则需要使用Python库才可以使用。minted和listings功能都非常强大,使用者需要根据自己的实际情况来选择与使用。
此处可以参考\href{https://www.latexstudio.net/archives/5900.html}{LaTeX之代码语法高亮}和\href{https://blog.csdn.net/xenonhu/article/details/88978672}{LaTeX中代码高亮宏包minted用法}
\end{frame}
\section{Title formats}

\begin{frame}{Metropolis titleformats}
	\themename 支持 4 种不同的标题格式:
	\begin{itemize}
		\item Regular
		\item \textsc{Smallcaps}
		\item \textsc{allsmallcaps}
		\item ALLCAPS
	\end{itemize}
	它们可以为每种标题类型一次性(全局)设置,也可以单独设置。
\end{frame}

\subsection{Tricks}

{
    \metroset{titleformat frame=smallcaps}
\begin{frame}{Small caps}
	这一帧使用 \texttt{smallcaps} 标题样式.

	\begin{alertblock}{潜在问题}
		请注意,并非每种字体都支持小型大写字母。 例如,如果您使用 pdfTeX 和 Computer Modern Sans Serif 字体排版演示文稿,则小型大写字母中的每个文本都将使用 Computer Modern Serif 字体排版。
	\end{alertblock}
\end{frame}
}

{
\metroset{titleformat frame=allsmallcaps}
\begin{frame}{All small caps}
	这一帧使用 \texttt{allsmallcaps} 标题样式.

	\begin{alertblock}{潜在问题}
		由于此标题格式也使用小型大写字母,因此您面临与使用 \texttt{smallcaps} 标题格式相同的问题。 此外,这种格式可能会导致一些其他问题。 如果您考虑使用它,请参阅文档。

		根据经验:只能将其用于纯文本标题。
	\end{alertblock}
\end{frame}
}

{
\metroset{titleformat frame=allcaps}
\begin{frame}{All caps}
	这一帧使用 \texttt{allcaps} 标题格式.

	\begin{alertblock}{潜在问题}
		这种标题格式不像 \texttt{allsmallcaps} 格式那样有问题,但基本上存在相同的缺陷。 因此,如果您想使用它,请查看文档。
	\end{alertblock}
\end{frame}
}

\section{Elements}

\begin{frame}[fragile]{排版}
      \begin{verbatim}
      	The theme provides sensible defaults to
\emph{emphasize} text, \alert{accent} parts
or show \textbf{bold} results.
\end{verbatim}

  \begin{center}
  	becomes
  \end{center}

  The theme provides sensible defaults to \emph{emphasize} text,
  \alert{accent} parts or show \textbf{bold} results.
\end{frame}

\begin{frame}{字体功能测试}
  \begin{itemize}
    \item Regular
    \item \textit{Italic}
    \item \textsc{SmallCaps}
    \item \textbf{Bold}
    \item \textbf{\textit{Bold Italic}}
    \item \textbf{\textsc{Bold SmallCaps}}
    \item \texttt{Monospace}
    \item \texttt{\textit{Monospace Italic}}
    \item \texttt{\textbf{Monospace Bold}}
    \item \texttt{\textbf{\textit{Monospace Bold Italic}}}
  \end{itemize}
\end{frame}

\begin{frame}{列表}
  \begin{columns}[T,onlytextwidth]
    \column{0.33\textwidth}
      Items
      \begin{itemize}
        \item Milk 
        \item Eggs 
        \item Potatos
      \end{itemize}

    \column{0.33\textwidth}
      Enumerations
      \begin{enumerate}
        \item First, 
        \item Second and 
        \item Last.
      \end{enumerate}

    \column{0.33\textwidth}
      Descriptions
      \begin{description}
        \item[PowerPoint] Meeh. 
        \item[Beamer] Yeeeha.
      \end{description}
  \end{columns}
\end{frame}

\begin{frame}{动态演示}
  \begin{itemize}[<+- | alert@+>]
    \item \alert<4>{This is\only<4>{ really} important}
    \item Now this
    \item And now this
  \end{itemize}
\end{frame}

\begin{frame}{Figures}
  \begin{figure}
    \newcounter{density}
    \setcounter{density}{20}
    \begin{tikzpicture}
      \def\couleur{alerted text.fg}
      \path[coordinate] (0,0)  coordinate(A)
                  ++( 90:5cm) coordinate(B)
                  ++(0:5cm) coordinate(C)
                  ++(-90:5cm) coordinate(D);
      \draw[fill=\couleur!\thedensity] (A) -- (B) -- (C) --(D) -- cycle;
      \foreach \x in {1,...,40}{%
          \pgfmathsetcounter{density}{\thedensity+20}
          \setcounter{density}{\thedensity}
          \path[coordinate] coordinate(X) at (A){};
          \path[coordinate] (A) -- (B) coordinate[pos=.10](A)
                              -- (C) coordinate[pos=.10](B)
                              -- (D) coordinate[pos=.10](C)
                              -- (X) coordinate[pos=.10](D);
          \draw[fill=\couleur!\thedensity] (A)--(B)--(C)-- (D) -- cycle;
      }
    \end{tikzpicture}
    \caption{Rotated square from
    \href{http://www.texample.net/tikz/examples/rotated-polygons/}{texample.net}.}
  \end{figure}
\end{frame}

\begin{frame}{Tables}
  \begin{table}
    \caption{Largest cities in the world (source: Wikipedia)}
    \begin{tabular}{lr}
      \toprule
      City & Population\\
      \midrule
      Mexico City & 20,116,842\\
      Shanghai & 19,210,000\\
      Peking & 15,796,450\\
      Istanbul & 14,160,467\\
      \bottomrule
    \end{tabular}
  \end{table}
\end{frame}

\begin{frame}{Blocks}
  三种不同的块环境是预定义的,并且可以使用可选的背景颜色设置样式。

  \begin{columns}[T,onlytextwidth]
    \column{0.5\textwidth}
      \begin{block}{Default}
        Block content.
      \end{block}

      \begin{alertblock}{Alert}
        Block content.
      \end{alertblock}

      \begin{exampleblock}{Example}
        Block content.
      \end{exampleblock}

    \column{0.5\textwidth}

      \metroset{block=fill}

      \begin{block}{Default}
        Block content.
      \end{block}

      \begin{alertblock}{Alert}
        Block content.
      \end{alertblock}

      \begin{exampleblock}{Example}
        Block content.
      \end{exampleblock}

  \end{columns}
\end{frame}

\section{\LaTeX{} Mathematical Symbols}
\subsection{Greek and Hebrew letters}
\begin{frame}[fragile]{Greek and Hebrew letters}
	\begin{table}
		\centering
		\caption{Greek and Hebrew letters}
		\begin{tabular}{cccc}
		$\alpha$ \verb|\alpha| & $\beta$ \verb|\beta| & $\chi$ \verb|\chi| & $\delta$ \verb|\delta| \\
		$\epsilon$ \verb|\epsilon| &
		 $\eta$ \verb|\eta| & $\gamma$ \verb|\gamma| & $\iota$ \verb|\iota| \\
		$\kappa$ \verb|\kappa| & $\lambda$ \verb|\lambda| &
		  $\mu$ \verb|\mu| & $\nu$ \verb|\nu| \\
		$o$ \verb|o| & $\omega$ \verb|\omega| & $\phi$ \verb|\phi| &
		   $pi$ \verb|\pi| \\
		$\psi$ \verb|\psi| & $\rho$ \verb|\rho| & $\sigma$ \verb|\sigma| & $\tau$ \verb|\tau| \\ 
		$\theta$ \verb|\theta| & $\upsilon$ \verb|\upsilon| & $\xi$ \verb|\xi| & $\zeta$ \verb|\zeta| \\
		$\digamma$ \verb|\digamma| & $\varepsilon$ \verb|\varepsilon| & $\varkappa$ \verb|\varkappa| & $\varphi$ \verb|\varphi| \\
		$\varpi$ \verb|\varpi| & $\varrho$ \verb|\varrho| &
		$\varsigma$ \verb|\varsigma| & $\vartheta$ \verb|\vartheta| \\
		$\Gamma$ \verb|\Gamma| & $\Lambda$ \verb|\Lambda| & $\Omega$ \verb|\Omega| & $\Phi$\verb|\Phi| \\
		$\Pi$ \verb|\Pi| & $\Psi$ \verb|\Psi| & $\Sigma$ \verb|\Sigma| & $\Upsilon$ \verb|\Upsilon| \\
		$\Xi$ \verb|\Xi| & $\aleph$ \verb|\aleph| & $\beth$ \verb|\beth| & $\daleth$ \verb|\daleth| \\
		$\gimel$ \verb|\gimel| & $\Delta$ \verb|\Delta|	& $\Theta$ \verb|\Theta| & \\
	\end{tabular}
	\end{table}
\end{frame}

\begin{frame}{Math}
	\begin{equation}
    e = \lim_{n\to \infty} \left(1 + \frac{1}{n}\right)^n 
	\end{equation}
	\begin{equation}
    \hat{f}(\xi):=\int_{-\infty}^{+\infty}f(x)e^{-2\pi i x \xi}{\rm d}(x)
	\end{equation}
\end{frame}


\begin{frame}{Line plots}
  \begin{figure}
    \begin{tikzpicture}
      \begin{axis}[
        mlineplot,
        width=0.9\textwidth,
        height=6cm,
      ]

        \addplot {sin(deg(x))};
        \addplot+[samples=100] {sin(deg(2*x))};

      \end{axis}
    \end{tikzpicture}
  \end{figure}
\end{frame}
\begin{frame}{Bar charts}
  \begin{figure}
    \begin{tikzpicture}
      \begin{axis}[
        mbarplot,
        xlabel={Foo},
        ylabel={Bar},
        width=0.9\textwidth,
        height=6cm,
      ]

      \addplot plot coordinates {(1, 20) (2, 25) (3, 22.4) (4, 12.4)};
      \addplot plot coordinates {(1, 18) (2, 24) (3, 23.5) (4, 13.2)};
      \addplot plot coordinates {(1, 10) (2, 19) (3, 25) (4, 15.2)};

      \legend{lorem, ipsum, dolor}

      \end{axis}
    \end{tikzpicture}
  \end{figure}
\end{frame}
\begin{frame}{Quotes}
  \begin{quote}
    Veni, Vidi, Vici
  \end{quote}
\end{frame}

{%
\setbeamertemplate{frame footer}{My custom footer}
\begin{frame}[fragile]{Frame footer}
    \themename defines a custom beamer template to add a text to the footer. It can be set via
    \begin{verbatim}\setbeamertemplate{frame footer}{My custom footer}\end{verbatim}
\end{frame}
}

\begin{frame}{References}
  Some references to showcase [allowframebreaks] \cite{knuth92,ConcreteMath,Simpson,Er01,greenwade93}
\end{frame}

\section{Conclusion}

\begin{frame}{Summary}

  Get the source of this theme and the demo presentation from

  \begin{center}\url{github.com/matze/mtheme}\end{center}

  The theme \emph{itself} is licensed under a
  \href{http://creativecommons.org/licenses/by-sa/4.0/}{Creative Commons
  Attribution-ShareAlike 4.0 International License}.

  \begin{center}\ccbysa\end{center}

\end{frame}

{\setbeamercolor{palette primary}{fg=black, bg=yellow}
\begin{frame}[standout]
  Questions?
\end{frame}
}

\appendix

\begin{frame}[fragile]{Backup slides}
  Sometimes, it is useful to add slides at the end of your presentation to
  refer to during audience questions.

  The best way to do this is to include the \verb|appendixnumberbeamer|
  package in your preamble and call \verb|\appendix| before your backup slides.

  \themename will automatically turn off slide numbering and progress bars for
  slides in the appendix.
\end{frame}

\begin{frame}[allowframebreaks]{References}

  \bibliography{demo}
  \bibliographystyle{abbrv}
  

\end{frame}

\end{document}
