\documentclass[8pt,UTF8]{ctexbeamer} 
%默认字体12pt,编码使用UTF8,使用中文
\usepackage[fontsize=8pt]{fontsize}

\usetheme[progressbar=frametitle]{metropolis} 
%使用metropolis这一beamer主题
\usepackage{appendixnumberbeamer}

\usepackage{booktabs} %书签
\usepackage[scale=2]{ccicons} %typeset Creative Commons icons

\usepackage{pgfplots}
\usepgfplotslibrary{dateplot}
\usepackage{hologo}

\usepackage{amsmath, amsfonts, amssymb} %数学符号

%语法高亮
\usepackage{minted}
\usepackage{listings}

\usepackage{}
\usepackage{hyperref}
\hypersetup{
	colorlinks=true,
	linkcolor=blue,
	filecolor=blue,      
	urlcolor=red,
	citecolor=cyan,
}

\usepackage{graphicx}
\DeclareGraphicsExtensions{.eps,.ps,.jpg,.bmp,.png,.svg} %指定可以使用以上格式图像文件

\usepackage{xspace} %断行命令
\newcommand{\themename}{\textbf{\textsc{metropolis}}\xspace}
\usepackage{lstlinebgrd}

\lstset{
	basicstyle=\ttfamily\tiny,
	frame=shadowbox,
	numbers=left,
	showstringspaces=false,
	showtabs=false,
	showspaces=false,
	keepspaces=true,
	backgroundcolor=\color{white},
	keywordstyle=\ttfamily\color[RGB]{40,40,255},
	emphstyle=\ttfamily\color[RGB]{40,40,255},
	numberstyle=\ttfamily\color{darkgray},
	commentstyle=\it\ttfamily\color[RGB]{0,96,96},
	stringstyle=\slshape\ttfamily\color[RGB]{128,0,0},
	language=python,
	breaklines=true,
	breakindent=30pt,
	tabsize=2,
	morekeywords={love,gettrace,settrace,parse,process,int32_t},
	linebackgroundcolor=\color{white},
}

\title{Metropolis} %标题
\subtitle{一个优秀的beamer主题} %副标题
\date{\today} %今天的日期
\author{李子奥 \hspace{2em} \href{mailto: leeziao0331@gmail.com}{leeziao0331@gmail.com}} %作者
% \author{Lee ZiAo \hspace{2em} \href{mailto: leeziao0331@gmail.com}{leeziao0331@gmail.com}} %作者
\institute{华中科技大学 \hspace{4em} 人工智能与自动化学院} %机构
% \institute{HuaZhong University of Science and Technology \hspace{4em} School of Artificial Intelligence and Automation} %机构
\titlegraphic{\hfill\includegraphics[height=2.5cm]{image/academic.png}} %插入logo

\newcommand{\iitem}[2]{\item \textbf{#1:} #2}
\newcommand{\headline}[1]{\centerline{\Large #1}}

%在每个section 前边单独插入当前章节高亮的目录页(当然最原始的目录页你还是需要手动录入的,不要想偷懒)
%\AtBeginSection[]
%{
%	\begin{frame}
%		\frametitle{当前章节}
%		\tableofcontents[currentsection]
%	\end{frame}
%}

\begin{document}

\maketitle %目录生成

\begin{frame}{目录}
	%		Table of contents 目录/大纲页
	%             自动实现对section的收集,并绘制成目录页
  \setbeamertemplate{section in toc}[sections numbered]
  \tableofcontents%[hideallsubsections]
\end{frame}

\begin{frame}{First Frame}
	\headline{Title of First Frame}
	\begin{itemize}
		\iitem{Item1}{Content1}
		\iitem{Item2}{Content2}
	\end{itemize}
\end{frame}

\begin{frame}[fragile]{Source Code Analysis}
  \begin{columns}[T,onlytextwidth]
    \column{0.5\textwidth}
	\changefontsize[8pt]{8pt}
	\lstinputlisting[language=c++,
		title=Source Code,
		linebackgroundcolor={%
			\ifnum\value{lstnumber}>5
				\ifnum\value{lstnumber}<14
					\color{green!35}
				\else
					\color{white}
				\fi
			\else
				\color{white}
			\fi}
	]{source.cpp}
    % \column{0.05\textwidth}
    \column{0.45\textwidth}
	\changefontsize[6pt]{6pt}
	\noindent 这个循环中有两个变量会在后续过程中用到:
	\begin{itemize}
		\item \texttt{i}:表示输入字符串的长度
		\item \texttt{quoteNum}:表示输入中 \small{Quote Character} 出现的次数
	\end{itemize}
	跳出循环的条件为:
	\begin{itemize}
		\item 遇到字符\texttt{'\textbackslash 0'}
		\item 最大循环次数\texttt{k}从\texttt{0xffff\_ffff}递减到0,表示已经遇到了\texttt{0xffff\_ffff}个单词(unicode字符)。
	\end{itemize}
	\pause
	\texttt{i}是一个32位的带符号变量,当输入字符串的长度大于\texttt{0x7fff\_ffff}时,\texttt{i}会发生溢出。
  \end{columns}
\end{frame}

\section{What is beamer?}
\begin{frame}[allowframebreaks]{What is beamer?}
	Beamer 是一个用于创建演示文稿 LaTeX 的文档类。它同时支持\LaTeX + dvips、\hologo{pdfLaTeX}、\hologo{LuaLaTeX}以及\hologo{XeLaTeX}。它的名称取自德语词汇 Beamer(pseudo-anglicism),意思是影像演示。
	
	Beamer文档类并不是最早开发出的\LaTeX 演示文稿工具。2003年2月,Till Tantau为其博士论文答辩编写了beamer包,并于一个月之后发布在CTAN上。
	
	作为LaTeX的一个文档类,Beamer文档和\LaTeX 文档一样都是纯文本文件。且beamer兼容\LaTeX 常见的命令,和其他宏包的兼容性良好。
	
	当然也有支持Beamer语法的图形界面,如AUCTEX和LyX。
	
	Beamer也可以通过使用兼容包来支持其他LaTeX演示文稿宏包的语法,包括 Prosper和Foils。
	
	Beamer默认生成PDF文件用于演示,其动态效果依靠创建多页幻灯片实现。
	
	若要打印出每张幻灯片的最终效果用于分发给听众,需开启handout选项;想要在一张纸上打印多页幻灯片,需要用pgfpages宏包;也可以输出适合印刷在A4或者标准信纸上的文档效果。
	
	'frame'的标题将变为段落的标题,不再包括原有的外观主题,同时保证了原有章节结构不被破坏——这就可以方便的输出演讲的提纲。
	
	Beamer的一些功能是依赖于PGF的。
	
	以上介绍来自WiKipedia。\footnote{https://zh.wikipedia.org/wiki/Beamer\_(LaTeX)}
	
	关于如何自制一份beamer,请浏览\href{https://zhuanlan.zhihu.com/p/423443762}{用LaTeX创建一个Beamer},可以关注专栏,及时获得推动与建议。
\end{frame}

\section{超链接}
\begin{frame}[fragile]{在beamer中设置超链接}
	可以这样
\begin{minted}{latex}
	\usepackage{hyperref}
	\hypersetup{
		colorlinks=true,
		linkcolor=blue,%%修改此处为你想要的颜色
		filecolor=blue,      
		urlcolor=red,
		citecolor=cyan,}
\end{minted}
也可以这样
\begin{minted}{latex}
	\usepackage[colorlinks,
	linkcolor=blue,       %%修改此处为你想要的颜色
	anchorcolor=blue,  
	citecolor=blue,       
	]{hyperref}
\end{minted}
更多请翻阅hyperref文档。
\end{frame}

\section{语法高亮}
\begin{frame}[fragile]{语法高亮的选择}
	\begin{itemize}
		\item 使用verbatim抄录环境:\verb|\verb|
		
		\item 使用minted环境:\mint{latex}{\mint}
		
		\item 使用listings环境:
		\begin{lstlisting}[language=c++]
#include <iostream>
int main()
		\end{lstlisting}
	\end{itemize}
其中verbatim是可以直接用的,minted和listings都需要在导言区加载宏包才可以使用,而minted则需要使用Python库才可以使用。minted和listings功能都非常强大,使用者需要根据自己的实际情况来选择与使用。
此处可以参考\href{https://www.latexstudio.net/archives/5900.html}{LaTeX之代码语法高亮}和\href{https://blog.csdn.net/xenonhu/article/details/88978672}{LaTeX中代码高亮宏包minted用法}
\end{frame}

\section{Elements}

\begin{frame}[fragile]{排版}
      \begin{verbatim}
      	The theme provides sensible defaults to
\emph{emphasize} text, \alert{accent} parts
or show \textbf{bold} results.
\end{verbatim}

  \begin{center}
  	becomes
  \end{center}

  The theme provides sensible defaults to \emph{emphasize} text,
  \alert{accent} parts or show \textbf{bold} results.
\end{frame}


\begin{frame}{列表}
  \begin{columns}[T,onlytextwidth]
    \column{0.33\textwidth}
      Items
      \begin{itemize}
        \item Milk 
        \item Eggs 
        \item Potatos
      \end{itemize}

    \column{0.33\textwidth}
      Enumerations
      \begin{enumerate}
        \item First, 
        \item Second and 
        \item Last.
      \end{enumerate}

    \column{0.33\textwidth}
      Descriptions
      \begin{description}
        \item[PowerPoint] Meeh. 
        \item[Beamer] Yeeeha.
      \end{description}
  \end{columns}
\end{frame}

\begin{frame}{Blocks}
  三种不同的块环境是预定义的,并且可以使用可选的背景颜色设置样式。

  \begin{columns}[T,onlytextwidth]
    \column{0.5\textwidth}
      \begin{block}{Default}
        Block content.
      \end{block}

      \begin{alertblock}{Alert}
        Block content.
      \end{alertblock}

      \begin{exampleblock}{Example}
        Block content.
      \end{exampleblock}

    \column{0.5\textwidth}

      \metroset{block=fill}

      \begin{block}{Default}
        Block content.
      \end{block}

      \begin{alertblock}{Alert}
        Block content.
      \end{alertblock}

      \begin{exampleblock}{Example}
        Block content.
      \end{exampleblock}

  \end{columns}
\end{frame}

\begin{frame}{Math}
	\begin{equation}
    e = \lim_{n\to \infty} \left(1 + \frac{1}{n}\right)^n 
	\end{equation}
	\begin{equation}
    \hat{f}(\xi):=\int_{-\infty}^{+\infty}f(x)e^{-2\pi i x \xi}{\rm d}(x)
	\end{equation}
\end{frame}


{%
\setbeamertemplate{frame footer}{My custom footer}
\begin{frame}[fragile]{Frame footer}
    \themename defines a custom beamer template to add a text to the footer. It can be set via
    \begin{verbatim}\setbeamertemplate{frame footer}{My custom footer}\end{verbatim}
\end{frame}
}

\begin{frame}{References}
  Some references to showcase [allowframebreaks] \cite{knuth92,ConcreteMath,Simpson,Er01,greenwade93}
\end{frame}

\section{Conclusion}

{\setbeamercolor{palette primary}{fg=black, bg=yellow}
\begin{frame}[standout]
  Questions?
\end{frame}
}

\appendix

\begin{frame}[allowframebreaks]{References}

  \bibliography{demo}
  \bibliographystyle{abbrv}

\end{frame}

\end{document}
