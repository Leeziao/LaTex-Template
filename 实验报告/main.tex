\documentclass[onecolumn,a4paper]{article}

\input{style/ch_xelatex.tex}
\input{style/scala.tex}

\lstset{frame=, basicstyle={\footnotesize\ttfamily}}


\usepackage{amsmath,amssymb}
\usepackage{lmodern}
\usepackage{iftex}

\graphicspath{ {images/} }
\usepackage{ctex}
\usepackage{multirow}
\usepackage{subfigure}
\usepackage{listings}
\lstset{
 columns=fixed,
 numbers=left,                                        % 在左侧显示行号
 numberstyle=\tiny\color{gray},                       % 设定行号格式
 frame=none,                                          % 不显示背景边框
 backgroundcolor=\color[RGB]{245,245,244},            % 设定背景颜色
 keywordstyle=\color[RGB]{40,40,255},                 % 设定关键字颜色
 numberstyle=\footnotesize\color{darkgray},
 commentstyle=\it\color[RGB]{0,96,96},                % 设置代码注释的格式
 stringstyle=\rmfamily\slshape\color[RGB]{128,0,0},   % 设置字符串格式
 showstringspaces=false,                              % 不显示字符串中的空格
 language={[x86masm]Assembler},                                        % 设置语言
}
%-----------------------------------------BEGIN DOC----------------------------------------

\begin{document}
% \renewcommand{\contentsname}{目\ 录}
% \renewcommand{\appendixname}{附录}
% \renewcommand{\appendixpagename}{附录}
% \renewcommand{\refname}{参考文献}
% \renewcommand{\figurename}{图}
% \renewcommand{\tablename}{表}
\renewcommand{\today}{\number\year 年 \number\month 月 \number\day 日}

\begin{titlepage}			%makes a title page. Remember to change the author, CID, username and group number to what is appropriate for you!
	\centering
	{\scshape\LARGE 华中科技大学\par}
	{\scshape \LARGE 人工智能与自动化学院\par}
	\vspace{1cm}
	\hrule height 2pt \vspace{1mm}\hrule height 1pt \vspace{0.4cm}
  {\huge\bfseries 微机原理实验n:\par}
  \vspace{0.4cm}\hrule height 1pt \vspace{1mm}\hrule height 2pt
	\vspace{1cm}
	{\Large\itshape 彭杨哲\par}		%remember to change these!
	\vspace{1cm}
	%		{\large Group \@group\unskip\strut\par}
% 	{\large Group: 42 \hfill CID: 00123456 \hfill Username: j.bloggs\par}		%remember to change these!
	{\large U201914634 \hfill\par}		%remember to change these!
	\vspace{1cm}
	{\large \today\par}
\end{titlepage}

\newpage

\section{实验目的}
\begin{itemize}
    \item 一
\end{itemize}

\section{实验内容}
\begin{itemize}
    \item 一
    \item 二
\end{itemize}

\section{程序框图}
如图所示


\section{实验步骤}
步骤步骤步骤

\section{重点知识}
\begin{enumerate}
    \item 一
    \item 二
    \item 三
\end{enumerate}

\section{参考程序}

\section{实验结果}
\subsection{单步运行各条指令并记录相关寄存器的数值}
\begin{lstlisting}
_STACK SEGMENT  	STACK;堆栈段开始
	       DW 100 DUP(?);定义100个不确定内容的字变量
_STACK 	ENDS 	 	 ;堆栈段结束
DATA SEGMENT;数据段开始
DATA 	ENDS ;数据段结束
CODE SEGMENT;代码段开始
\end{lstlisting}
\subsection{详细注释每一条指令的功能}
\begin{lstlisting}
_STACK SEGMENT  	STACK;堆栈段开始
	       DW 100 DUP(?);定义100个不确定内容的字变量
_STACK 	ENDS 	 	 ;堆栈段结束
\end{lstlisting}

\section{思考题}
\begin{enumerate}
\item 问题\\
答: 答案
\end{enumerate}

\end{document}
