\documentclass[aspectratio=169, 10pt, utf8, mathserif]{beamer}
%导言区
\usepackage{ctex}
\usepackage{amsmath, amsfonts, amssymb, amsthm}
\usepackage{graphicx}
\usepackage{fontspec}
\usepackage{ulem} %解决下划线换行紊乱
\usepackage{caption} %添加图、表的标题
\usepackage{subfigure}
\usepackage{theorem}
\usepackage[backend=bibtex,sorting=none]{biblatex} %不列出所有作者
%\usepackage[backend=bibtex,sorting=none,maxnames=9,minnames=3]{biblatex} %列出所有作者,具体选择列不列可以由其前的“%”来决定
\addbibresource{ref.bib} %BibTeX数据文件及位置
\setbeamerfont{footnote}{size=\tiny} %设置脚注引用文献的字体大小
%\setbeamertemplate{bibliography item}[text] %设置参考文献图标样式数字标号
\usepackage{appendix} %增加附录
\usepackage{multicol} %分栏
\usepackage{syntonly} %只编译文件是否成功,省时省力
%\syntaxonly %不注释代表只编译是否成功
%\usepackage[marginal]{footmisc} %首页添加脚注无缩进
%\renewcommand{\thefootnote}{} %首页添加脚注无编号
\usepackage{enumerate}
\usepackage{listings} %代码包
\usepackage{xcolor} %代码高亮包
\lstset{
 language=Matlab, %代码语言使用的是matlab
 %frame=shadowbox, %把代码用带有阴影的框圈起来
 %rulesepcolor=\color{red!20!green!20!blue!20}, %代码块边框为淡青色
 keywordstyle=\color{blue}\bfseries, %代码关键字的颜色为蓝色,粗体
 commentstyle=\color{orange}\ttfamily, %设置代码注释的颜色,原字体样式\textit
 backgroundcolor=\color{darkgray!6}, %背景色
 showstringspaces=false, %不显示代码字符串中间的空格标记
 numbers=left, %显示行号
 numberstyle=\tiny, %行号字体
 basicstyle=\ttfamily,
 stringstyle=\ttfamily, %代码字符串的特殊格式
 breaklines=true, %过长的代码自动换行
 extendedchars=false,  %解决代码跨页时,章节标题,页眉等汉字不显示的问题
 escapebegin=\begin{CJK*}{GBK}{hei},escapeend=\end{CJK*} %防止中文报错
 texcl=true,
 morekeywords={classdef,function,global,parfor,persistent,spmd,plot}} %设置更多关键词

%使用的主题样式和主题色
\usetheme{Antibes}
\usecolortheme{beaver}
\usefonttheme{serif} %已有的字体default professionalfonts serif structurebold structureitalicserif structuresmallcapsserif

% 设置用acrobat打开就会全屏显示
\hypersetup{pdfpagemode=FullScreen}

% 设置logo
% \pgfdeclareimage[height=0.5cm]{university-logo}{logo-name} %需提前将logo文件放到`.tex`文件中。
% \logo{\pgfuseimage{university-logo}}

%-------------开始-------------------
\begin{document}

%每个章节都有小目录
\AtBeginSection[]
{
 \begin{frame}
 \frametitle{章节目录}
 \begin{multicols}{2}
  \tableofcontents[currentsection]
 \end{multicols}
 \end{frame}
}

\title{Beamer 模板}
\subtitle{利用已有主题实现自己的主题}
\author[姓名]{李子奥 \\ \zihao{6}{\href{mailto:183xxxxxx0@163.com}{leeziao0331@gmail.com}}
   \quad \\ \vspace{0.5cm} 导师姓名\quad\zihao{6}{导师信息}}
\institute[学校地区,学校名称]
{
 实验室名称 \\
 专业及方向
}
\date{\today}
%显示封面页
\begin{frame}
    %\maketitle
    \titlepage
\end{frame}

\begin{frame}
 \frametitle{总目录}
 \begin{multicols}{2}
  \tableofcontents[hideallsubsections]
 \end{multicols}
 %\tableofcontents[hideallsubsections]
\end{frame}

\section{使用已有主题的方法}
\subsection{主题样式颜色}
\begin{frame}
 \frametitle{使用已有主题的方法}
可以直接点击该链接\underline{\href{https://mpetroff.net/files/beamer-theme-matrix/}{已有的主题样式和主题颜色}}。横栏表示主题颜色,纵栏表示主题样式。

将想套用的主题样式和颜色放到usetheme\{Szeged\}和usecolortheme\{beaver\}中即可。
\end{frame}

\section{公式及编号}
\subsection{带编号的公式}
\begin{frame}
 \frametitle{带编号的公式}
现在展示一个带编号的公式:
 \begin{equation}
 f(x) = \frac{\mathrm e^{2x}}{\sin x}
 \end{equation}
\end{frame}

\subsection{不带编号的公式}
\begin{frame}
 \frametitle{不带编号的公式}
 另外再展示一个不带编号的公式。
\[
\mathrm e^{\mathrm i \pi} + 1 = 0 
\]
\end{frame}

\subsection{行内公式}
\begin{frame}
 \frametitle{行内公式}
 以及一个行内公式$a^2 + b^2 = c^2$.
\end{frame}

\section{列表环境}
\subsection{无序列表和逐条展示的功能}
\begin{frame}
 \frametitle{列表}
这是无序列表的样式,及逐条展示的功能。
 \begin{itemize}
 \item 无序列表标号1
 \pause
 \item 无序列表标号2
 \end{itemize}
\end{frame}

\subsection{有序列表}
\begin{frame}
 \frametitle{有序列表}
这是有序列表的样式及一次性的逐条展示功能。
 \begin{enumerate}[<+-|alert@+>]
 \item 这是1
 \item 这是2
 \end{enumerate}
\end{frame}

\section{块环境}
\subsection{放某些特定的句子和公式}
\begin{frame}
 \frametitle{块环境}
 \begin{exampleblock}{Beamer介绍}
 Beamer是\LaTeX 的一个文档类,主要用于学术报告幻灯片的制作,优点是跨平台性好,支持Windows,Mac等。导出的格式就是PDF。
 \end{exampleblock}
 \begin{alertblock}{不同样式的文本块}
 \begin{equation}
  \left \{
  \begin{aligned}
  f(x) &= 2x + b \\
  g(x) &= x + 9
  \end{aligned} 
  \right.
  \end{equation}
 \end{alertblock}
 \begin{block}{Beamer介绍}
 \begin{equation}
 \left \{
 \begin{aligned}
 f(x) &= 2x + b \\
 g(x) &= x + 9
 \end{aligned} 
 \right.
 \end{equation}
 \end{block}

\end{frame}


\section{代码环境}
\begin{frame}[fragile] %必须加[fragile]
 \frametitle{MATLAB代码}  
 \begin{lstlisting}[numbers=left, firstnumber=753]
  % 绘制图形
  x = 1 : 0.01 : 5;
  y = sin(x);
  plot(x, y)
 \end{lstlisting}
\end{frame}

\section{致谢}
\begin{frame}
 \zihao{-4}\centering{坚持学习,不是为了输赢。}
\end{frame}
\end{document}